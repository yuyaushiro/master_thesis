\chapter*{}\addcontentsline{toc}{chapter}{要旨}

\section*{要旨}
本論文では, 状態推定の不確かさを考慮した障害物回避行動をロボットに行わせるための手法を提案する. 
実世界において, 認識の不確かさは不可避の問題だが, 人間や一部の生物は不確かさに応じて適切に行動する. 
ロボットにこのような知的な行動をさせるためには, 行動決定アルゴリズムで情報の不確かさを考慮することが必要である. 
現在多くの移動ロボットにおいて, 姿勢推定には確率論に基づく不確かさが考慮された推定が取り入れられている一方で, 行動決定には不確かさが考慮されていない. 
姿勢推定では信念分布と呼ばれる確率分布で姿勢を表現するが, 行動決定には最も確率の高い姿勢のみを利用するため, 分布の大きさや形は動作に反映されない. 
この問題に対し, 信念分布自体を行動決定に用いることで, 不確かさを考慮した行動をロボットに行わせる研究が複数存在する. 
しかし, 移動ロボットにおいて重要な, 障害物回避行動に対し不確かさを考慮させる研究は見られない. 
提案手法は, MCLで自己位置推定を行う移動ロボットに対し, パーティクル全体が障害物を回避するような動作を行わせることで, 信念分布内のロボットが障害物に衝突するのを防ぐ. 
障害物付近のパーティクルが行動決定に与える影響を一時的に大きくすることで, この動作を実現する. 
障害物を回避しゴールへと向かうナビゲーションタスクにおける動作を, 従来手法と比較することで, 提案手法の有効性が確認された. 

\section*{キーワード}
自律移動ロボット, POMDPs, probabilistic flow control method
