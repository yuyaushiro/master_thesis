\chapter{不確かさを考慮した障害物の回避動作} \label{chapter:mechod}
本章では、不確かさを考慮した障害物の回避動作を生成する手法について述べる。
\ref{section:method overview}節では、提案手法の概要について簡単に述べる。
\ref{section:価値関数}節では、観測の不確かさがない前提で価値関数を計算する方法について述べる。
\ref{section:障害物}節では、本論文において回避を行う障害物の定義を述べる。
\ref{section:回避重み}節では、提案手法においてPFC法から変更する部分について述べる。

\newpage


%%%%%%%%%%%%%%%%%%%%%%%%%%%%%%%%%%%%%%%%%%%%%%%%%%%%%%%%%%%%%%%%%%%%%%%%%%%%%%%%
\section{提案手法の概要} \label{section:method overview}
本論文では、不確かさを考慮した障害物の回避行動を生成する。
MCLにより自己位置推定を行う移動ロボットの行動決定に、パーティクルの分布をそのまま利用し、推定の不確かさを考慮した行動を行わせる。
本手法では、ロボットの信念分布を近似するパーティクルの分布全体が、障害物外を移動するようにロボットを動作させる。
これにより、ロボットの自己位置推定が正しく行われており、真の姿勢がパーティクルの分布内に存在する場合、
ロボットが障害物に侵入するのを防ぐことができる。

%----------------序論に持ってくるべきかもしれない------------------
% 自己位置推定にMCLを利用する現在の多くの移動ロボットは、最も重みの大きいパーティクルを自身の真の姿勢と仮定する。
% そして、その姿勢が目的地へと向かうために必要な行動をとることで、自律移動を行っている。
% 自己位置推定の不確かさ極めて小さいときは有効であるが、不確かさが大きいときには問題が発生する可能性がある。

% たとえば、ロボットが図\ref{fig:example task}に示すような経路を移動する場合について考える。
% 図\ref{fig:}に示すように

\ref{section:PFC法}節で述べたPFC法に変更を加えることで、この動作を実現する。
オフラインフェーズでは、通常のPFC法同様に、観測に関する不確かさがない前提でMDP問題を解き、最適価値関数を計算する。
変更は、実際にロボットの行動決定を行うオンラインフェーズでの行動評価の式に行う。
PFC方は、全パーティクルが行動決定に与える影響を同じにするのではなく、ゴール付近のパーティクル影響を強めている。
この影響度を、障害物付近のパーティクルでも強くする。
次ステップの行動で障害物内に侵入する可能性があるパーティクルが、行動決定に与える影響度を一気に大きくすることで、回避動作を行わせる。
これにより、パーティクルの分布全体が障害物を避けるような動作を行わせるとともに、PFC法の利点である、ゴール付近で分布をゴールに流し込むような探索動作を行うことが可能となる。


%%%%%%%%%%%%%%%%%%%%%%%%%%%%%%%%%%%%%%%%%%%%%%%%%%%%%%%%%%%%%%%%%%%%%%%%%%%%%%%%
\section{価値反復による価値関数の計算} \label{section:価値関数}


%%%%%%%%%%%%%%%%%%%%%%%%%%%%%%%%%%%%%%%%%%%%%%%%%%%%%%%%%%%%%%%%%%%%%%%%%%%%%%%%
\section{本論文における障害物} \label{section:障害物}


%%%%%%%%%%%%%%%%%%%%%%%%%%%%%%%%%%%%%%%%%%%%%%%%%%%%%%%%%%%%%%%%%%%%%%%%%%%%%%%%
\section{回避重みの導入} \label{section:回避重み}
