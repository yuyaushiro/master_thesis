\chapter{結論} \label{chapter:conclusion}

%%%%%%%%%%%%%%%%%%%%%%%%%%%%%%%%%%%%%%%%%%%%%%%%%%%%%%%%%%%%%%%%%%%%%%%%%%%%%%%%
\section{本論文の結論}
本論文では、自己位置推定の不確かさを考慮した障害物の回避行動を自律移動ロボットに行わせるための手法を提案した。
MCLによって自己位置推定を行うロボット用に、信念分布を近似するパーティクルの分布全体が、障害物を回避するような動作を生成した。
この動作は、行動決定に与える影響度を決定するパラメータを各パーティクルに持たせ、この数値をタスク実行中に動的に変化させることで実現した。
次の行動で障害物に侵入するパーティクルの数値を大きくし、障害物を避けようとする動作を強めることで、目的の動作を達成した。

また、提案手法の有効性について、先行研究と比較し評価した。
既存の手法では、障害物内をパーティクルの分布が移動することにより、ロボットの障害物への衝突が起こる可能性があることを示した。
提案手法は、パーティクルの分布が障害物内を移動するのをなくしたことで、この衝突の可能性を大幅に減らすことに成功した。

他にも、本論文で提案した手法では、ロボットの挙動に無駄があることが問題と述べた。
移動ロボットが障害物を回避するとき、ギザギザの軌跡を描くように動作する。
これは、パーティクル一つが障害物に入っただけで、動作が大きく変わりすぎるためだと考えられる。

\newpage

%%%%%%%%%%%%%%%%%%%%%%%%%%%%%%%%%%%%%%%%%%%%%%%%%%%%%%%%%%%%%%%%%%%%%%%%%%%%%%%%
\section{今後の展望}
今後、ロボットの挙動の無駄の多さについて改善する必要があると言える。
これは、今回新たに導入した、パーティクルが行動決定に与える影響度の増やし方を工夫することで実現できる可能性がある。
現在の手法では、パーティクルが一つでも障害物に侵入しそうになった場合、影響度を一気に最大にしている。
これを徐々に増加させていくように変更する方法が一つ考えられる。
また、パーティクルの数から、次の行動でロボットが障害物に衝突する確率が計算できる。
この確率に応じて、ロボットの行動を徐々に回避動作へと移行させるように、パラメータを増加させていくことも、有効であると考えられる。

今回の評価実験のような、ゴールと障害物が遠くに存在している環境では、
障害物の回避を行うフェーズとゴール探索動作を行うフェーズが分かれている。
しかし、ゴール付近に障害物が存在する場合、障害物回避動作とゴール探索動作はトレードオフとなり、
2つの動作をループするようなデッドロックが発生する可能性があると考えられる。
そういった場合、障害物への衝突確率を判断材料に、行動を回避動作から探索動作へ徐々に移行させていくような方法が有効ではないかと考えられる。
また、パラメータの最大値も各パーティクルに個別で持たせることで、
障害物に入りそうになる度に、行動へ与える影響度が大きくなりにくくなるような方法も、有効ではないかと考えられる。
